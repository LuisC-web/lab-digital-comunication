\documentclass[10pt]{article}

% Idioma y codificación
\usepackage[utf8]{inputenc}
\usepackage[T1]{fontenc}

% Paquete para matemáticas avanzadas
\usepackage{amsmath}

\title{Fórmulas comunicaciones digitales}

\date{\today}

\begin{document}
	
	\maketitle
	
\section{Fórmulas para todos los casos}
\begin{enumerate}
 	\item Tasa de transferencia de simbolo
 	\begin{equation}
 		R_s=\frac{1}{T_s}[\#simbolos/tiempo]
 	\end{equation}
	\item Cantidad de simbolos
	\begin{equation}
		M=2^{\eta_b} [simbolos]
	\end{equation}
	\item  Tasa de transmisión por usuario
	\begin{equation}
		R_U=\frac{R_T}{\eta_c} [bits/simb]
	\end{equation}
	Donde $\eta_c$ es numeros de los canales y $R_T$  es tranmisión total
	\item Frecuencia de muestreo
	\begin{equation}
	f_{mt}>2f_{max}
	\end{equation}
	Unidad [$\frac{1}{s}=Hz$]
	\item  Probabilidad del simbolo
	\begin{equation}
		P(si)=\frac{\# de simbolo}{\# de simbolos totales}
	\end{equation}
	
	\item  Medida de información
		\begin{equation}
		I(s_i)=\log_{n}{\frac{1}{P(si)}} [medida]
	\end{equation}
		Donde $n$ es la base que de acuerdo a la medida de información es:
	
	\begin{enumerate}
		\item   n=2; Se refiere a la unidad bit
		\item   n=10; Se refiere a la unidad Hartley
		\item   n=e; Se refiere a la Nat
	\end{enumerate} 
	Cada uno de los anterior se refier al la medida por ejemplo 2 bits, 2 Nats, 3 Hartley. Solo se refiere a la unidad de medición
	\item Factor de conversion de una medida de información $m->n$
	\begin{equation}
		\log_{n}{\frac{1}{P(si)}}=\log_{m}{\frac{1}{P(si)}}log_{n}{m}
	\end{equation}
	donde $\log_{m}{\frac{1}{P(si)}}$ seria la medida con la unidad actual. Por ejemplo si se tiene $3bits$ y se quiere pasar a nat
	\begin{equation}
		\ln{\frac{1}{P(si)}}=3\log_e{2}=2.079 nats
	\end{equation}
		También se puede usar la siguiente
	\begin{equation}
	I(s_i)=\log_{2}{M} [bit]
\end{equation}

	

	
	\item Propiedad del logaritmo
		\begin{equation}
		\log_a{b}=\frac{log_c{b}}{log_c{a}}
	\end{equation}
	\item Entropía
	\begin{equation}
	 	H(s)=\sum_{i=0}^{M-1}I(si)P(si)[bit/simb]
	\end{equation}
	Donde $I(si)$ es la cantidad de información por simbolo que se mide en nat, bit o hartley y $P(si)$ es la probabilidad del simbolo
	\item Cantidad de simbolo por mensaje
	\begin{equation}
	\frac{\#simb}{msj}=R_sT_{msj}[\#simb/msj]
	\end{equation}
	\item Valor promedio por mensaje
	\begin{equation}
	 \overline{I}=\frac{\#simb}{msj}H(s)[bits/msj]
	\end{equation}
	 \item  Logintud media
	 \begin{equation}
	 	L=\sum P(xi)L_i[bit/palabra]
	 \end{equation}
	 Donde $L_i$ es la longitud de la palabra o simbolo y  $P(xi)$ probabilidad del simbolo o palabra
	 \item  Señales binarias arrojadas al canal
	 
	 \begin{equation}
	 	S_{arrj}=R_sL
	 \end{equation}
\end{enumerate}

\section{Para sistemas AM}
Consideración para AM
\begin{enumerate}
	\item Banda de transmisión
	\begin{equation}
		BW_{tx}=2B
	\end{equation}
\end{enumerate}


\subsection{PWM}
	\begin{enumerate}
		\item Ecuación general
		\begin{equation}
			\tau[s]=b[s]+k[V]f(t)[s/V]
		\end{equation}
		Donde las unidades de $\tau$ es tiempo(s), las unidades de b igual y $f(t)$ son $s/V$ y k está en Voltios(V)
		\item Por lo general se tiene que
		\begin{equation}
			\tau_g=\tau_o=T_m*5\%
		\end{equation}
		Donde $\tau_g$ es el tiempo de guarda y $\tau_o$
		\item Resolución
		\begin{equation}
		\Delta\tau=k_f(V_R)
		\end{equation}
		\item Ancho de banda de transmisión
		\begin{equation}
			Bw_{tx}=\frac{1}{10\Delta\tau}
		\end{equation}
	\end{enumerate}
	\section{x-ASFK}
	Donde x es cualquier sistema 2ASFK,4ASFK y etc
	
	\begin{enumerate}
			\item Banda sumprimida
		\begin{equation}
			B'=\frac{C.C}{T_s}
		\end{equation}
		\item  Periodo de transferencia de simbolo para un fitro ideal
		\begin{equation}
			T_s=\frac{1}{2B_{filtro}}
		\end{equation}
		\item Transferencia de simbolos
		\begin{equation}
			R(si)=\frac{I(si)}{T_s}[bit/s]
		\end{equation}
		\item Casos dependiendo la situación
		\begin{enumerate}
			\item  Cuando no hay filtro
			\begin{equation}
				BW_{tx}=2B
			\end{equation}
			\item  Cuando el filtro es ideal
			\begin{equation}
				BW_{tx}=2B'
			\end{equation}
			Donde B' el ancho de banda suprimido por el filtro

			\item Cuando el filtro real
		\begin{equation}
				\gamma =\frac{f_a}{f_c}
			\end{equation}
			Donde $\gamma$ es el fator de Rolloff o de caída
			\begin{equation}
				\frac{1}{T_s} =\frac{2B_{filtro}}{1+\gamma}
			\end{equation}
		
			\end{enumerate}
	
	\end{enumerate}
	
\section{Ejercicios}
\subsection{4ASK}
Se transmite un tren de pulso en un 4ASK y se tiene que $R(si)=2Mbit/s$, $M=4$ obtenga $BW_{tx}$ para cada caso:
\begin{enumerate}
	\item Para un tramisión sin filtro
	\begin{equation}
		BW_{tx}=2B=2\infty=\infty
	\end{equation}

		\item Para un filtro ideal y un $B_{filtro}$=2C.C
		\begin{equation}
		I(si)=\log_{2}{4}=2
		\end{equation}
		Se sabe que
		\begin{equation}
			T_s=\frac{I(si)}{R}=\frac{2}{R}
		\end{equation}
		Se calcula $B'$
		\begin{equation}
			B'=\frac{2}{T_s}=\frac{2R}{2}=R
		\end{equation}
		Se calcula $Bw_{tx}$
		\begin{equation}
			Bw_{tx}=2B'=2(2Mbit/s)=4MHz
		\end{equation}
		\item Con $\gamma=0.5$ y un filtro real
			\begin{equation}
			I(si)=\log_{2}{4}=2
		\end{equation}
		Se sabe que
		\begin{equation}
			T_s=\frac{I(si)}{R}=\frac{2}{R}
		\end{equation}
		Se sabe que para un filtro real
		\begin{equation}
			\frac{1}{T_s}=\frac{2Bw_{filtroo}}{1+\gamma}=R
		\end{equation}
		Se calcula $Bw_{filtro}$
		\begin{equation}
		\frac{R}{2}=2B'=\frac{2Bw_{filtroo}}{1+\gamma}
		\end{equation}
		\begin{equation}
			Bw_{filtro}=\frac{R(1+\gamma)}{4}=\frac{2Mbit/s*(1+0.5)}{4}=750kHz
		\end{equation}
		Se sabe que $Bw_{filtro}=B'$ por lo que
		\begin{equation}
			Bw_{tx}=2B'=1.5MHz
		\end{equation}
\end{enumerate}
\section{x-FSK}
\begin{equation}
	Bw_{tx}=2B'+2\Delta f
\end{equation}
\end{document}
